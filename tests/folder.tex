%!TEX program = lualatex
\documentclass{article}
\usepackage{specimen}
\begin{document}
% Vogel Quax zwickt Johnys Pferd Bim.
% Sylvia wagt quick den Jux bei Pforzheim.
% Prall vom Whisky flog Quax den Jet zu Bruch.
% „Oh, welch Zynismus!“, quiekte Xavers jadegrüne Bratpfanne.
% Jeder wackere Bayer vertilgt bequem zwo Pfund Kalbshaxen.
% Franz jagt im komplett verwahrlosten Taxi quer durch Bayern.
% Stanleys Expeditionszug quer durch Afrika wird von jedermann bewundert.

% „Üb jodeln, Gör!“ quäkt Schwyz’ Vamp fix.
% „Ja!“ quäkt Schwyz’ Pöbel fix vor Gmünd.
% Schwyz’ Vamp quäkt öd: „Fix, lob Jürgen!“
% Ob Schwyz’ Vamp dünkt „Fix, quäle Jörg“?

% Vogt Nyx: „Büß du ja zwölf Qirsch, Kämpe!“
% Verbüß öd’ Joch, kämpf Qual, zwing Styx!
% „Fix, Schwyz!“ quäkt Jürgen blöd vom Paß.
% Jux-Typ aß schäbig vor Zwölf-qkm-Düne
% Jörg bäckt quasi zwei Haxenfüße vom Wildpony
% Welch fieser Katzentyp quält da süße Vögel bloß zum Jux?
% Baß verwünscht jäh Xylophon, Querflöte und ganze Musik.
% Züchtigsüß rät Quasidame bösem Pony: „Wirf Klavier zum Jux“.
% Falsches Üben von Xylophonmusik quält jeden größeren Zwerg.
% Zwölf Boxkämpfer jagen Eva quer über den großen Sylter Deich.
% Polyfon zwitschernd aßen Mäxchens Vögel Rüben, Joghurt und Quark.
% Schweißgequält zündet Typograf Jakob verflixt öde Pangramme an.
% Vom Ödipuskomplex maßlos gequält, übt Wilfried zyklisches Jodeln.
% Asynchrone Bassklänge vom Jazzquintett sind nix für spießige Löwen
% Xaver schreibt für Wikipedia zum Spaß quälend lang über Yoga, Soja und Öko.
% Die heiße Zypernsonne quälte Max und Victoria ja böse auf dem Weg bis zur Küste.
% Zornig und gequält rügen jeweils Pontifex und Volk die maßlose bischöfliche Hybris.
% Vögel üben Gezwitscher oft ähnlich packend wie Jupp die Maus auf dem Xylophon einer Qualle.

\sampletext{Franz jagt im komplett verwahrlosten Taxi quer durch B}
% /Users/jf/Library/texmf/fonts/opentype/itc
% /Users/jf/Library/texmf/fonts/opentype/adobe/pro/warnock
%\fontsampler{/Users/jf/Library/texmf/fonts/opentype/monotype}

\fontsampler{/usr/share/fonts/truetype/liberation}


\end{document}